\documentclass[handout]{beamer}
\usetheme[progressbar=frametitle]{metropolis}
\usepackage{appendixnumberbeamer}
\usepackage{booktabs}
\usepackage{amsmath}
\usepackage{amssymb}
\usepackage{tcolorbox}
\definecolor{metropolisblue}{RGB}{39, 59, 94}
\usepackage{xcolor}

% Define custom colors
\definecolor{myblue}{HTML}{007AFF}
\definecolor{mygreen}{HTML}{4CD964}
\definecolor{myred}{HTML}{FF3B30}
\definecolor{myorange}{HTML}{FF9500}

\begin{document}

\newcommand{\uncertaintyTable}{\begin{table}
        \begin{tabular}{|r|c|l|r|c|}
            \hline
            Age & Gender & Height & FVC & Respiratory health \\
            \hline
            21  & M      & 1.60   & 5.2 & Good               \\
            35  & M      & 1.50   & 4.8 & OK                 \\
            15  & F      & 1.70   & 3.2 & Good               \\
            45  & M      & 1.60   & 2.8 & Poor               \\
            \hline
            9   & F      & 1.0    & 2.2 & ?                  \\
            \hline
        \end{tabular}
    \end{table}}

\begin{frame}{Uncertainty applications}
    We can not afford to be overconfident in medical applications.

    \uncertaintyTable

\end{frame}

\begin{frame}{Uncertainty applications}
    We can not afford to be overconfident in medical applications.

    \uncertaintyTable

    \begin{itemize}
        \item Deterministic model: ``Poor'' with 90\% probability.
    \end{itemize}

\end{frame}

\begin{frame}{Uncertainty applications}
    We can not afford to be overconfident in medical applications.

    \uncertaintyTable

    \begin{itemize}
        \item Deterministic model: ``Poor'' with 90\% probability.
        \item Probabilistic model: Since the point is OOD, ``Good'' with 30\% probability, ``OK'' with 30\% probability, ``Poor'' with 40\% probability.
    \end{itemize}

\end{frame}

\newcommand{\StudyQ}{Q: You took a course at IITGN where final exam has very high weightage. You gave the exam and it didn't go well. What will be your grade?\\
    Options: (1) 10, (2) 9, (3) 8, (4) 7}

\begin{frame}{Grade}
    \StudyQ
\end{frame}

\begin{frame}{Grade}
    \StudyQ \\
    \hfill \\
    Hint: In the previous year, the median grade was 9.
\end{frame}

\end{document}