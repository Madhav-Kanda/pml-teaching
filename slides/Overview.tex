\documentclass[handout]{beamer}

\usetheme[progressbar=frametitle]{metropolis}
\usepackage{appendixnumberbeamer}
\usepackage{booktabs}
\usepackage{amsmath}
\usepackage{amssymb}
\usepackage{tcolorbox}
\usepackage{pgfplots}
% compatibility of pgfplots
\pgfplotsset{compat=1.16}
\usepackage{tikz}
\usepackage{pgf}
\definecolor{metropolisblue}{RGB}{39, 59, 94}



% Begin document
\begin{document}

% Title page
\title{Overview}
\author{Nipun Batra}
\date{\today}
\institute{IIT Gandhinagar}
\maketitle


\begin{frame}
    \begin{enumerate}
        \item Why Bayesian ML?
        \item Bayesian Inference basics: Bayes Rule in the ML context
        \item Probability Refresher
        \item Maximum Likelihood Estimation: Distributions, Linear Regression, Logistic Regression, NN
        \item Aleatoric and Epistemic Uncertainty
        \item Maximum A Posteriori Estimation: Distributions, Linear Regression, Logistic Regression, NN
        \item Conjugate Priors: Distributions, Linear Regression (BLR)
    \end{enumerate}
\end{frame}

\begin{frame}
    % Continue counting from previous slide

    \begin{enumerate}
        \item Bayesian Logistic Regression: Laplace Approximation
        \item predictive distribution: Linear Regression (closed form), Logistic Regression (Laplace Approximation + probit)
        \item Monte Carlo Methods: General, Value of Pi, etc, predictive distribution for linear and logistic regression
        \item MCMC: Metropolis Hastings, Gibbs Sampling, Hamiltonian Monte Carlo
        \item Information Theory: KL Divergence, Cross Entropy, Mutual Information
        \item Variational Inference
        \item Bayesian Neural Networks: MC Dropout, Deep Ensembles
        \item Gaussian Processes
        \item Neural Processes
        \item Active Learning
        \item Bayesian Optimization
        \item Variational Autoencoders
    \end{enumerate}
    
\end{frame}

\begin{frame}{Projects Category I: Tools}
    \begin{enumerate}
        \item RMH in Hamiltorch
        \item 
    \end{enumerate}

    
\end{frame}

\end{document}